\chapterimage{Predicciones-del-furturo-del-retail-de-moda-para-el-2020.jpg}
\chapter{Trabajos Futuros}
A partir del lanzamiento del Sistema Gestión de Monitorias, se plantea trabajar en el desarrollo de nuevas versiones del programa, la cuales cuenten con la opción de correr el aplicativo en un dispositivo móvil como una aplicación nativa, para mejorar la experiencia del usuario, así mismo se podrá acceder desde cualquier dispositivo con conexión a Internet por medio de un sitio web. Adicionalmente se buscará dar la opción a los usuarios de digitalizar los documentos que sean necesarios con la aplicación. El objetivo final de la aplicación es sustituir todo el papeleo utilizado en el proceso de monitorias, desde la entrega de los documento que se hace al inicio del proceso, hasta la cancelación de los pagos a los monitores.
\newline
El Sistema Gestión de Monitorias será extendido a los demás proyectos curriculares, los cuales cuenten con este servicio de monitorias. Esto implicaría adaptar el sistema a los diferentes requerimientos que surgen en los diferentes proyectos curriculares de la universidad y posiblemente de diferentes universidades.  
\newline
El tener un sistema que logre sustituir el papeleo físico de un proceso administrativo abre las puertas al desarrollo de diferentes aplicativos que tengan una finalidad similar, con lo cual se aportaría a disminuir notablemente el papeleo utilizado y sería un paso importante para aprovechar, aún más, los recursos tecnológicos dentro de la universidad Distrital Francisco José de Caldas.