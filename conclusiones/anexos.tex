\chapterimage{addb4870c144badba811c06724df8512.jpg}
\chapter{Anexos}

\lstset{language=Java,
	showspaces=false,
	showtabs=false,
	breaklines=true,
	showstringspaces=false,
	breakatwhitespace=true,
	commentstyle=\color{pgreen},
	keywordstyle=\color{pblue},
	stringstyle=\color{pred},
	basicstyle=\ttfamily,
	moredelim=[il][\textcolor{pgrey}]{$$},
	moredelim=[is][\textcolor{pgrey}]{\%\%}{\%\%}
}

\section{Código}


\subsection{Patrón Singleton}

\headline{\textbf{Clase abstracta Base de Datos}} 
\lstinputlisting{CodigoPatrones/Proxy/BaseDatos.java}
\begin{figure}[H]
	\centering
	\includegraphics[width=0.8\linewidth]{codSingleton1}
	\centering
	\caption{Clase BDSQLite, parte del patrón proxy, con singleton implementado}
	\label{fig:codSingleton1}
\end{figure}
\begin{figure}[H]
	\centering
	\includegraphics[width=0.8\linewidth]{codSingleton2}
	\centering
	\caption{Clase ProxyBD, parte del patrón proxy, con singleton implementado}
	\label{fig:codSingleton2}
\end{figure}
\subsection{Patrón Fachada}


\headline{\textbf{Interfaz Imonitoria}} 
\lstinputlisting{CodigoPatrones/Fachada/IMonitoria.java}
\headline{\textbf{Clase Horas Monitoria}}
\lstinputlisting{CodigoPatrones/Fachada/HorasMonitoria.java}
\headline{\textbf{Clase Clasificados}}
\lstinputlisting{CodigoPatrones/Fachada/Clasificados.java}
\headline{\textbf{Cliente}}
\lstinputlisting{CodigoPatrones/Fachada/Cliente.java}

\subsection{Patrón Componente}
\begin{figure}[H]
	
	\includegraphics[width=0.5\linewidth]{codComposite}
	
	\caption{Clase AClasificado, contiene interfaz que implementan los clasificados}
	\label{fig:codSingleton2}
\end{figure}
\clearpage
\begin{figure}[H]
	\centering
	\includegraphics[width=1\linewidth]{codComposite1}
	\centering
	\caption{Clase Bloque, contiene los objetos tipo clasificados}
	\label{fig:codSingleton2}
\end{figure}
\begin{figure}[H]
	\centering
	\includegraphics[width=0.8\linewidth]{codComposite2}
	\centering
	\caption{Clase Clasificado}
	\label{fig:codSingleton2}
\end{figure}
\subsection{Patrón Proxy}

\headline{\textbf{Clase abstracta Base de Datos}} 
\lstinputlisting{CodigoPatrones/Proxy/BaseDatos.java}
\headline{\textbf{Clase concreta BDSQLite}} 
\lstinputlisting{CodigoPatrones/Proxy/BDSQLite.java}
\headline{\textbf{Clase proxy Base de Datos}} 
\lstinputlisting{CodigoPatrones/Proxy/ProxyBD.java}
\headline{\textbf{Cliente}} 
\lstinputlisting{CodigoPatrones/Proxy/Cliente.java}


\subsection{Patrón Comando}
\begin{figure}[H]

	\includegraphics[width=0.8\linewidth]{codComando1}

	\caption{Clase Cliente que ejecuta el comando}
	\label{fig:codComando1}
\end{figure}
\begin{figure}[H]

	\includegraphics[width=0.6\linewidth]{codComando3}

	\caption{Interface comando}
	\label{fig:codComando3}
	
\end{figure}
\clearpage
\begin{figure}[H]

	\includegraphics[width=0.8\linewidth]{codComando2}

	\caption{Clase Aplicacion donde se manejan las consultas a la BD}
	\label{fig:codComando2}
\end{figure}

\begin{figure}[H]
	\centering
	\caption{Implementación de la interfaz comando por cada metodo de Aplicacion}
	\includegraphics[width=1\linewidth]{codComando4}
	\caption{Comando de Modificación}
	\centering
	\label{fig:codComando4}
\end{figure}
\clearpage
\begin{figure}[H]
	\centering
	\includegraphics[width=1\linewidth]{codComando5}
	\centering
	\caption{Comando de Adición}
	\label{fig:codComando5}
\end{figure}

\begin{figure}[H]
	\centering
	\includegraphics[width=1\linewidth]{codComando6}
	\centering
	\caption{Comando de Eliminación}
	\label{fig:codComando6}
\end{figure}
\clearpage
\begin{figure}[H]
	\centering
	\includegraphics[width=1\linewidth]{codComando7}
	\centering
	\caption{Comando de Consulta}
	\label{fig:codComando7}
\end{figure}

\begin{figure}[H]
	\centering
	\includegraphics[width=1\linewidth]{codComando8}
	\centering
	\caption{Clase Menu con la que interactua el cliente en la GUI}
	\label{fig:codComando8}
\end{figure}

