\chapterimage{ayuda-psicologica-920x460.jpg} 
\chapter{Problema}
%%Este es el problema \cite{Bol_2012,Bol_2016}
\begin{figure}[th!]
	\centering
	\includegraphics[width=0.5\linewidth]{Personas.png}
\end{figure}

\section{Introducción}

Las monitorías en la Universidad Distrital, son un incentivo que ofrece la institución a estudiantes de todas
las carreras a cambio de una serie de beneficios. Actualmente la forma en la que se controla no es la adecuada, lo cual genera una serie de trabas en la entrega de incentivos. A continuación se especificará en detalle el problema y la solución que se propone.

\newpage
\section{Definición}
La universidad Distrital Francisco José de Caldas al igual que algunas otras universidades ofrece estímulos a sus estudiantes por medio de convocatorias tales como la de los asistentes académicos e investigativos, a los cuales se les denomina comúnmente como monitores. Las labores de los monitores son diversas, pero las más comunes son el apoyo al docente y el estudiante. Son los encargados de apoyar diferentes tareas dentro de la academia y deben trabajar 12 horas semanales durante el periodo académico en el que se postulan. Los beneficios obtenidos a la hora de ser monitor van desde 2 salarios mínimos legales vigentes hasta un descuento de la mitad del valor de una especialización. 
\newline
\newline
El estudiante espera que dichos incentivos sean entregados lo más pronto posible apenas acabada su labor, sin embargo el proceso de entrega para estos tienen una serie de trabas por parte de la universidad, hasta tal punto que el estudiante es recompensado hasta 4 meses después de culminada su tarea. El problema generalmente radica en el incumplimiento por parte de algunos monitores en el desempeño de sus labores, ello retrasa por un buen tiempo el pago de las nóminas.
\newline
\newline
Pero no todo es culpa del estudiante, algunas veces es simplemente culpa de la planeación, el control y los docentes. Todo monitor cuenta con un horario fijo en el cual deben cumplir sus labores, el estudiante supone que cumplirá sus 12 horas semanales durante las 16 semanas de estudio, pero esto no sucede así. Durante algunas semanas dadas las circunstancias el monitor se atrasa en el cumplimiento de sus tareas, como por ejemplo cuando el docente falla sin previa notificación, esto genera un coste de oportunidad para el estudiante pues no solo pierde su valioso tiempo, sino que también deberá recuperar una inasistencia de la cual no es responsable.
\newline
\newline
Al final de la última semana de estudios es cuando se esperaría que todos los monitores hicieran entrega de sus planillas de asistencia, sin embargo es en esa semana donde el estudiante tiene que rebuscarse dentro de la academia para lograr completar sus horas, aquellos monitores responsables lo consiguen en  el último día, mientras que los otros sencillamente no le prestan importancia y afectan el proceso para todos los demás. Dado estos hechos aquí mencionados se hace necesario la implementación de un control más riguroso y automatizado que dé vía libre a un procedimiento más rápido en la entrega de incentivos para los monitores.
\newpage
\newpage
\section{Solución}
Para lograr que el proceso de las monitorias sea más eficiente y dar solución a los problemas más comunes que se encuentran a la hora de asignar monitores, se quiere sistematizar la información de manera que sea más organizado este proceso, de igual forma se busca facilitar el acercamiento entre profesores y estudiantes, con los monitores que estén disponibles, lo cual ayuda a que estos últimos puedan completar sus horas laborales y garantiza que cumplan con su deber, para de esta forma recibir los incentivos de forma oportuna. Se quiere reducir de forma significativa la perdida de horas por parte de los monitores debido a algún imprevisto que se presente con el profesor, como por ejemplo, que este no pueda asistir a clase y que un monitor encuentre diferentes alternativas para recuperar estas horas.
\newline
\newline
Con el fin de lograr lo anteriormente planteado, se propone desarrollar una aplicación híbrida, con la que se pueda gestionar las actividades realizadas por los monitores y tanto las horas cumplidas, como las que estén aún pendientes, así como la certificación de estas por parte de los profesores. Con la aplicación se busca dar un control a las horas de trabajo realizadas, ya que al terminar una sesión el profesor podrá certificarla por medio de la aplicación, con lo que se controla que los monitores cumplan con sus labores establecidas. En caso de que el profesor no asista a la clase o que no necesite del monitor, este podrá buscar un profesor que si requiera uno, esto se hará a través de un módulo que contiene a los monitores libres en ese momento, los cuales podrán ser vistos por los profesores quienes tendrán la opción de contactarlos y con esto evitar que el monitor pierda tiempo, igualmente ayudar para que complete las horas en las que estaba atrasado o que adelante tiempo de trabajo, siempre que un profesor certifique su labor.
\newline
\newline
Uno de los objetivos es organizar la información, de forma que se pueda eliminar el papeleo que se emplea en este proceso y que sea más fácil acceder a ella, por medio de la aplicación se podrá manejar toda la información referente a los profesores, los horarios, los monitores y las horas laboradas por estos. Toda la información será registrada en una base de datos y las personas autorizadas podrán consultarla para comprobar que los procesos se estén realizando con normalidad, los estudiantes tendrán la opción de llevar un control de su tiempo de forma más sencilla y práctica.
\newpage
\section{Objetivos}
\subsection{Objetivo general}
Desarrollar un software de plataforma web, en el cual se realizara la automatización del sistema de monitorias con el fin de apoyar la organización y control en la Universidad Distrital Francisco José de Caldas.
\subsection{Objetivos específicos}
\begin{itemize}
\item Realizar un diseño inicial de la aplicación mediante técnicas y modelos lógicos de tal manera que se pueda realizar un esbozo de la aplicación de manera coherente y eficiente.
\end{itemize}
\begin{itemize}
\item Diseñar una interfaz accesible para administradores mediante la cual se pueda llevar un permanente control de los estudiantes que pertenecen al programa de monitorias de la universidad.
\end{itemize}
\begin{itemize}
\item Mejorar la oportunidad de completar y verificar las horas que los estudiantes tienen que cumplir  durante el semestre y con ello constatar su desempeño durante el semestre.
\end{itemize}

\newpage
\section{Justificación}
El presente proyecto de software aquí expuesto buscará dar solución a los retrasos de las nóminas de los monitores de la Universidad Distrital, por medio de una automatización parcial del proceso logrando así beneficiar a cientos de estudiantes que ejecutan sus labores como asistentes académicos e investigativos.
\subsection{Justificación social}
Llevar a cabo una aceleración del proceso de las monitorías beneficiaría más que todo a la comunidad académica, pues permitiría que los pagos se efectuaran casi inmediatamente luego de finalizada las correspondientes labores de tal forma que los estudiantes puedan invertir su dinero en nuevos proyectos sin tener que esperar largas trabas del proceso.
\subsection{Justificación tecnológica}
Un proyecto de esta índole no se había llevado a cabo dentro de la universidad, lo cual hace que el producto sea innovador dentro de este aspecto. Por otro lado la automatización del proceso y la digitalización de la documentación requerida para la certificación de las labores, permitirá ahorrar tiempo y recursos.
\subsection{Justificación ambiental}
El hecho de digitalizar algunos componentes del proceso como las planillas de asistencia, documentos del estudiante, horarios y entre otros, no significan un gran impacto en beneficio al medio ambiente. Sin embargo a largo plazo, al usar menos hojas de papel y tinta se contribuye al cuidado del planeta y la generación de menos residuos.